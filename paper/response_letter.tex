\documentclass[12pt]{letter}
\usepackage[left=1in,twoside=false]{geometry}
\usepackage[cmex10]{amsmath}
\usepackage{amssymb}
\signature{Jinesh C Jacob}
\address{From \\
Mohit Madan\\
Department of Electrical  Engineering\\
Indian Institute of Technology Bombay, Mumbai\\
India\\
E-mail: mohit.madan@iitb.ac.in
\date{\today}
}

\begin{document}
\begin{letter}{To\\
Prof. Xiao Li\\
Editor\\
IEEE Wireless Communications Letters}
\vspace{0.5in}


\opening{Dear Prof. Li,}

Subject: \textbf{Revised submission of manuscript title ``Scalar
  Feedback based Joint Time-Frequency Precoder interpolation for  MIMO-OFDM Systems'' (Paper Number: WCL2020-0046).}

Authors: Mohit Madan, Agrim Gupta, Kumar Appaiah\\
Email: mohit.madan@iitb.ac.in

Thank you for arranging the prompt reviews of our original manuscript
submitted to the IEEE Wireless Communications Letters, and providing
us an opportunity to resubmit this manuscript with modifications
addressing the issues raised during the reviews. I would like to thank
the reviewers for a careful reading of the manuscript, as well as for
the insightful points that they have raised regarding various
technical aspects of the manuscript. Based on these reviews, we have
made extensive additions and revisions to our submission to address
the reviewers concerns. In particular, we have addressed the key issue
raised by reviewer 2 as well as the Editor, regarding the important
feature of our method that provides us better performance using scalar
quantization. We have also addressed all typos to the best of our
ability. We hope that you will reconsider the manuscript for
publication in the IEEE Wireless Communications Letters. Please find
attached our comments and actions taken based on the reviews, along
with a document indicative of all the changes made.

Please do not hesitate to contact me if I may be of any assistance; I
can be reached by e-mail at mohit.madan@iitb.ac.in. Thank you again
for your consideration of this manuscript.


Thank you.
\vspace{0.3in}

Sincerely,

Mohit Madan



\end{letter}

\newpage

\textbf{Editor comments}\\

\textbf{There are several mistakes and typos in the manuscript, as
  also pointed out by both reviewers. Moreover, it is mentioned that
  the temporal correlation based and frequency correlation based
  precoder tracking have been discussed in [8] and [9]. Thus the
  differences between the proposed method and the methods in [8] and
  [9] should be explained and compared, so as to make the
  contributions more clear.}

\textbf{Response:} Thank you for pointing this out. The temporal
correlation based precoder tracking discussed in [8] and [9] have
considered about precoder tracking in time and frequency respectively,
albeit separately, without any joint interpolation.  We have
approached this differently by performing prediction in the time
domain and using this for joint interpolation on the time-freqeuncy
grid. This helps build a framework for reducing the total bit
bandwidth by a signficant amount while maintaining similar performance
in terms of BER.  We have altered the text in the introduction to make
this point more clear,  as follows:

``\emph{In our work, we perform prediction in time domain, and use this for joint
  interpolation on the time frequency grid.}''

\textbf{Reviewer 1}\\

\textbf{This paper presents an interesting case study of quantized
  precoder design with limited feedback channels. The results show
  that quantization base on scalar parameterization is more effective
  than direct quantization on manifolds. There are, however, two
  issues that need to be addressed before publication.}

\textbf{1. It is not clear whether the feedback bandwidth budgets are
equivalent for methods in comparison. The comparison must be fair.}

\textbf{Response:} Thank you for pointing this out. We have used an
average of 5 bits in our method. While the number of feedback bits
used in [2] is 6 bits per OFDM frame, the approach that we suggest
uses a smaller number bits on average while still achieving similar or
better BER performance. We have now added a brief indication of this
in the text as follows

\emph{Since the bit budget is small, if the channel varies
sufficiently slowly, the $\theta_{k,l}$ and the $\phi_{k,l}$ parameters
are fed back alternately, while retaining the unsent parameters as
their previous values, as shown in
Fig. 1. This brings down the bit budget to an average of 5 bits for
$\bf{\Theta}$ and 5 bits for $\bf{\Phi}$ over alternate feedback
instances, thereby resulting in an average feedback budget of 5 bits,
as opposed to the 6 bits of feedback used
in [2].}

\textbf{ The so-called time-frequency interpolation method is just a
  straightforward two-dim filter. So I assume the that the more
  important factor for performance improvement would be the use of an
  efficient parameterization for the precoder matrix, which comes from
  ref. [8]. I would expect this paper shed more light on why this
  particular parameterization is more efficient in characterizing
  precoder matrices than the description based on manifolds.  Would
  the authors make more explanations/comments on this?}

\textbf{Response:}

Thank you for the feedback. The interpolation approach that we have
proposed is simple and ends up becoming more accurate over time when
compared to the manifold based interpolation that uses complex
interpolation methods in the manifold space. It is also easier to
track the adaptation of the scalar parameters using few bits. As
observed in Fig. 2, the quantization error obtained when using the
parameter based approach taken in this manuscript is much smaller in
that obtained using the conventional codebooks on the Stiefel
manifold, as discussed in [2]. This results in higher accuracy as
compared to manifolds with similar number of bits.

To emphasize these points, we have made the following addition:

\emph{With smaller
quantization error as compared to manifold based approaches, the
feedback results in more accurate precoder tracking, along with simple
2D-interpolation over the scalar parameters to optimize the precoders.}

\textbf{There are some typos/errors to be fixed:}

\textbf{1. Work/works got mixed up for many times.}

Thank you for this comment. We have fixed all instances of this error.

\textbf{2. ``The quantization approach described above... '' Actually,
  parameterization is described without presenting the quantizing
  method.}

\textbf{Response:} We thank the reviewer for pointing out this
error. We have altered the text to include this:

\emph{The distribution of $\theta_{k,l}$ is used for
quantizing $\bf{\Theta}$ during initialization to reduce the
quantization error.  If $n$ bits are allocated to quantize a
parameter for the first time, the range of values that each parameter
takes is divided into $2^n$ optimal intervals using Lloyd's
quantization [17].}

\textbf{3. "While tracking the $\phi_i$ parameters, is must be noted... "}

\textbf{Response:} It has been corrected to \emph{``While tracking the $\phi_i$ parameters, it must be noted... ''}.

\textbf{4. "we will use precoder parameters from both the past and future and
time instances to do the interpolation... "}

\textbf{Response:} Thank you. The typo has now been corrected to read
``we will use precoder parameters from both the past and future time instances to do the interpolation... ''

\textbf{5. "the parameters that determine the precoder is ...  = 10, ... "}

\textbf{Response:} Thanks for pointing this out. It has been corrected to ``\emph{...the parameters that determine the precoder is
$N_{R}(2N_{T} - N_R) = 12$, with five $\theta_{k,l}$s and and seven
$\phi_{k,l}$s.}''

\textbf{Reviewer 2}\\

\textbf{1. There are several mistakes and typos in the manuscript. Here are some examples.}

\textbf{(a) In the manuscript, boldface letters are used to denote column vectors and matrices and non-bold letters are used to denote scalars. Therefore, the subscripts of $\boldsymbol{\eta_{i,t}}$, $\boldsymbol{\Theta_{i,t}}$ and $\boldsymbol{\Phi_{i,t}}$ should be non-bold, and the covariance matrix of $\boldsymbol{\eta}$, should be boldface.}

\textbf{Response:} Thanks for pointing out this typo. The subscripts
of the symbols have been corrected wherever they have incurred. In addition,
the covariance symbol have now been made bold.

\textbf{(b) Because the matrix $\tilde{V_{i,t}}$ is $N_T$ by $N_R$ and
  the vector $x_{i,t}$ is $N_d$ by 1, the product $V_{i,t}x_{i,t}$ in
  (1) is not well defined}.

\textbf{Response:} Thanks for pointing this out.  It now reads
\emph{We consider the case where $N_R < N_T$, and the data vector size
  $N_d = N_R$} which makes $x_{i,t}$ dimensions as $N_R$ which now
makes it well defined.

\textbf{(c) Because the matrix $I_{N_R}$ is $N_R$ by $N_R$ and the
  matrix $0_{N_T , N_T - N_R }$ is $N_T$ by $N_T - N_R$, the matrix
  $\tilde{I}$ = [$I_{N_R}, 0_{N_T, N_T - N_R }$] , in (2) is not
  well-defined}

\textbf{Response:} Thanks for pointing this out.  We have now fixed
this to read:

``\emph{and the $N_T \times N_R$ matrix
  $\tilde{\textbf{I}}$ =
  $\big[\textbf{I}_{N_R }, {\boldsymbol{0}}_{N_R ,N_T
    -N_R}\big]^{T}$}''

\textbf{(d) The right-hand side of (3) is a function of t but the left-hand side of (3) is independent of t.}

\textbf{Response:} ``t'' has been changed to ``$N_T$''. This makes the
right-hand side a square matrix of size $N_T \times N_T$ which is
indeed the size of the matrix $G_{m-1,m}\big(\theta\big)$ on the
left-hand side.

\textbf{(e) Do $\theta_l$ and $\phi_k$ represent the same things as $\theta_{k,l}$ and $\phi_{k,l}$ , respectively.}

\textbf{Response:} Thanks for pointing this out. Yes, $\theta_l$ and
$\phi_k$ represent the same things as $\theta_{k,l}$ and
$\phi_{k,l}$. Therefore, those references have now been corrected
wherever they appeared. In addition, we have also specified the pdf of
the $\theta_{k,l}$ clearly.

\textbf{(f) It is mentioned in Subsection III.A that the number of
  $\phi_k$ parameters is $N_R(2N_T - N_R -1 )/2$ while the number of
  $\theta$ parameters is $N_R(2N_T -N_R + 1)/2$." However, in the
  provided example with $N_T$ = 4 and $N_R$ = 2, there are 7
  $\phi_{k,l}$ 's and 5 $\theta_{k,l}$'s.}

\textbf{Response:} There was an error and we had interchanged the
expression for the number of parameters. Therefore, we have now
corrected it as follows:

``\emph{The number of $\phi_{k,l}$ parameters is
  $N_{R}(2N_{T} - N_{R}+1)/2$ while the number of $\theta_{k,l}$
  parameters is $N_{R}(2N_{T} - N_{R}-1)/2$.}''

\textbf{(g) It is mentioned in Subsection III.B that where $x_n$ is the unquantized value and $\tilde{x}_{n-1}$ is the unquantized (accurate) value." What does this mean?}

\textbf{Response:} We have now corrected this to ``\emph{where $x_n$ is
  the unquantized (accurate) value and $\tilde{x}_{n-1}$ is the
  quantized value}''

\textbf{(h) It is mentioned in Subsection III.B that where M is the increment used during the positive and negative moves." However, M is not an increment in (5).}

\textbf{Response:} Thanks for pointing this out. $M$ is the scaling
factor, and we have now corrected the reference to $M$ as follows:

``\emph{...where M is the scaling factor used during the positive and...}''


\textbf{(i) (Section IV) "The channel is simulated using the typical Jakes Model for for the COST207 channel power delay profiles." Here is a typo.}

\textbf{Response:} It has now been corrected to ``\emph{The channel is simulated using the typical Jakes Model for the COST207 channel power delay profiles.}''

\textbf{(j) (Section IV) the parameters that determine the precoder is $N_R (2N_T - 1) - N^{2}_R$ = 10, with five $\theta_l$s and and seven $\phi_i$s." Here is a typo because 10 $\neq$ 5 + 7.}

\textbf{Response:} Thanks for pointing out this typo. The formula for
calculating the precoder has now been corrected to ``\emph{
  $N_{R}(2N_{T} - N_R) = 12$ , with five $\theta_{k,l}$s and and seven
  $\phi_{k,l}$s.}''

\textbf{(k) What does the sentence "The time based predictive approaches correspond to temporal prediction using the manifold as well as scalar parameter based approaches without utilizing the joint time frequency correlations." in Section IV mean?}

\textbf{Response:} Thanks for pointing out this error. The time based
predictive approach was meant to imply the independent time-frequency based
approach. The text in the paper has been corrected as follows:


``\emph{Independent time-frequency based approaches correspond to the
  method without joint-time frequency interpolation i.e. with
  prediction in time domain along with separate interpolation in the frequency
  domain.}''

\textbf{2. The quantization of $\phi_{k,l}$'s and $\theta_{k,l}$'s mentioned in Subsection III.A and the interpolation
process mentioned in Subsection III.C should be explained more clearly.}

\textbf{Response:}
Thanks for pointing this out. The explanation of the quantization
process has now been made more  clear by altering the text in III.A as
follows:

``\emph{If $n$ bits are allocated to quantize a parameter
for the first time, the range of values that each parameter takes is
divided into $2^n$ optimal intervals using Lloyd's
quantization [17].}''\\

In addition, Section III.C is also altered as follows to make it more
clear:

\emph{``By
  future time instances it means that the precoder estimation is
  improved at a later time instant i.e.  after receiving the feedback
  of the next set of subcarriers.''}

\textbf{3. As mentioned in Fig. 1 and Subsection III.C, the future
  precoder parameters are required to conduct the interpolation. How
  can we obtain the future precoder parameters?}

\textbf{Response:} AGRIM TODO

We have altered the text in the subsection III.C as follows:
``By future time instances it means that the precoder estimation is improved
at a later time instant i.e. after receiving the feedback of the next set of
subcarriers.''

\textbf{4. Only 10 channel realizations are used to evaluate the BER characteristics. Is it enough?}

Thank you for this comment. Although we simulate over 10 realizations,
the channel evolves beginning with each realization, resulting in
sufficient averaging over various channel characteristics. Therefore,
the results that we present represent simulation over a sufficiently
large number of bits transmitted over enough channel variations. We
have now altered the text in the simulation section to reflect this as
follows:

\emph{The BER achieved by the proposed predictive quantization based
  precoder is compared against the ideal precoder for 100 channel
  evolutions in each run and averaged over 10 channel realizations
  (translating to averaging over $125,000$ bits for 64-point DFT and
  $2\times 10^6$ bits for the 1024-point DFT case). The error
  properties obtained by simulating transmission over these varying
  channels is sufficient to obtain good average performance
  properties.}

\textbf{5. In the manuscript, the time-frequency correlation is used for precoder tracking. Because it is mentioned in Section I that the precoder tracking have been discussed in [8] (based on temporal correlation) and [9] (based on frequency correlation), the difference among the proposed method and the methods in [8] and [9] should be explained and compared in the manuscript.}

\textbf{Response:} Thanks for pointing this out.  First of all the
interpolation is simple and accurate as compared to manifold based
interpolation which uses complex interpolation methods in the manifold
space. It is also easier to track the scalar parameters with fewer
bits as compared to codebook based method which cannot track its
parameters with such high accuracy. As mentioned in figure 2, the
quantization error is much smaller in given's based parametrization
which results in higher accuracy as compared to manifolds with similar
number of bits.

Additions have been made in the conclusion section as follows `` \emph{With lesser quantization error compared to manifold based method, the
errors in feedback are reduced. Moreover, they are further
handled easily by using a simple 2d-interpolation over the scalar parameters.}''


\textbf{6. The reference format is not consistent and there are several typos. The title of the manuscript is not in IEEE format.}
\end{document}
