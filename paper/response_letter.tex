\documentclass[12pt]{letter}
\usepackage[left=1in,twoside=false]{geometry}
\usepackage[cmex10]{amsmath}
\usepackage{amssymb}
\signature{Jinesh C Jacob}
\address{From \\
Mohit Madan\\
Department of Electrical  Engineering\\
Indian Institute of Technology Bombay, Mumbai\\
India\\
E-mail: mohit.madan@iitb.ac.in
\date{\today}
}

\begin{document}
\begin{letter}{To\\
Prof. Xiao Li\\
Editor\\
IEEE Wireless Communications Letters}
\vspace{0.5in}


\opening{Dear Prof. Li,}

Subject: \textbf{Revised submission of manuscript title ``Scalar
  Feedback based Joint Time-Frequency Precoder interpolation for  MIMO-OFDM Systems'' (Paper Number: WCL2020-0046).}

Authors: Mohit Madan, Agrim Gupta, Kumar Appaiah\\
Email: mohit.madan@iitb.ac.in

Thank you for arranging the prompt reviews of our original manuscript
submitted to the IEEE Wireless Communications Letters, and providing
us an opportunity to resubmit this manuscript with modifications
addressing the issues raised during the reviews. I would like to thank
the reviewers for a careful reading of the manuscript, as well as for
the insightful points that they have raised regarding various
technical aspects of the manuscript. Based on these reviews, we have
made extensive additions and revisions to our submission to address
the reviewers concerns. In particular, we have addressed the key issue
raised by reviewer 2 as well as the Editor, regarding the important
feature of our method that provides us better performance using scalar
quantization. We have also addressed all typos to the best of our
ability. We hope that you will reconsider the manuscript for
publication in the IEEE Wireless Communications Letters. Please find
attached our comments and actions taken based on the reviews, along
with a document indicative of all the changes made.

Please do not hesitate to contact me if I may be of any assistance; I
can be reached by e-mail at mohit.madan@iitb.ac.in. Thank you again
for your consideration of this manuscript.


Thank you.
\vspace{0.3in}

Sincerely,

Mohit Madan



\end{letter}

\newpage

\textbf{Editor comments}\\

\textbf{There are several mistakes and typos in the manuscript, as
  also pointed out by both reviewers. Moreover, it is mentioned that
  the temporal correlation based and frequency correlation based
  precoder tracking have been discussed in [8] and [9]. Thus the
  differences between the proposed method and the methods in [8] and
  [9] should be explained and compared, so as to make the
  contributions more clear.}

\textbf{Response:} Response here

\textbf{Reviewer 1}\\

\textbf{This paper presents an interesting case study of quantized
  precoder design with limited feedback channels. The results show
  that quantization base on scalar parameterization is more effective
  than direct quantization on manifolds. There are, however, two
  issues that need to be addressed before publication.}

\textbf{1. It is not clear whether the feedback bandwidth budgets are
equivalent for methods in comparison. The comparison must be fair.}

\textbf{Response:}

\textbf{ The so-called time-frequency interpolation method is just a
  straightforward two-dim filter. So I assume the that the more
  important factor for performance improvement would be the use of an
  efficient parameterization for the precoder matrix, which comes from
  ref. [8]. I would expect this paper shed more light on why this
  particular parameterization is more efficient in characterizing
  precoder matrices than the description based on manifolds.  Would
  the authors make more explanations/comments on this?}

\textbf{Response:}

\textbf{There are some typos/errors to be fixed:}

\textbf{1. Work/works got mixed up for many times.}

\textbf{2. "The quantization approach described above... " Actually,
  parameterization is described without presenting the quantizing
  method.}

\textbf{3. "While tracking the $\phi_i$ parameters, is must be noted... "}

\textbf{4. "we will use precoder parameters from both the past and future and
time instances to do the interpolation... "}

\textbf{5. "the parameters that determine the precoder is ...  = 10, ... "}

\textbf{Response:}
\end{document}
