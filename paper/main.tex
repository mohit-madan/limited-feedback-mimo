\documentclass[journal,10pt,twocolumn]{IEEEtran}
\def\b0{{\bf 0}}
\def\ba{{\bf a}}
\def\bb{{\bf b}}
\def\bc{{\bf c}}
\def\bd{{\bf d}}
\def\be{{\bf e}}
\def\bg{{\bf g}}
\def\bh{{\bf h}}
\def\bi{{\bf i}}
\def\bj{{\bf j}}
\def\bk{{\bf k}}
\def\bl{{\bf l}}
\def\bm{{\bf m}}
\def\bn{{\bf n}}
\def\bo{{\bf o}}
\def\bp{{\bf p}}
\def\bq{{\bf q}}
\def\br{{\bf r}}
\def\bs{{\bf s}}
\def\bt{{\bf t}}
\def\bu{{\bf u}}
\def\bv{{\bf v}}
\def\bw{{\bf w}}
\def\bx{{\bf x}}
\def\by{{\bf y}}
\def\bz{{\bf z}}


\def\bSigma{{\bf \Sigma}}
\def\bA{{\bf A}}
\def\bB{{\bf B}}
\def\bC{{\bf C}}
\def\bD{{\bf D}}
\def\bE{{\bf E}}
\def\bF{{\bf F}}
\def\bG{{\bf G}}
\def\bH{{\bf H}}
\def\bI{{\bf I}}
\def\bJ{{\bf J}}
\def\bK{{\bf K}}
\def\bL{{\bf L}}
\def\bM{{\bf M}}
\def\bN{{\bf N}}
\def\bO{{\bf O}}
\def\bP{{\bf P}}
\def\bQ{{\bf Q}}
\def\bR{{\bf R}}
\def\bS{{\bf S}}
\def\bT{{\bf T}}
\def\bU{{\bf U}}
\def\bV{{\bf V}}
\def\bW{{\bf W}}
\def\bX{{\bf X}}
\def\bY{{\bf Y}}
\def\bZ{{\bf Z}}

\setlength{\textfloatsep}{0.7\baselineskip plus 0.2\baselineskip minus 0.5\baselineskip}
\def\imwidth{0.62}
%package list
\IEEEoverridecommandlockouts
\let\labelindent\relax
% \usepackage{breqn}
\usepackage{enumitem}
\usepackage{fleqn}
\usepackage{cite}
\usepackage{graphicx}
\usepackage[varg]{newtxmath}
\graphicspath{ {images/} }
\usepackage{pdfpages}
\usepackage{wrapfig}
\usepackage{fancyhdr}
\usepackage{lastpage}
\usepackage{lettrine}
\usepackage{amsmath}
% \usepackage{titlesec}
\usepackage[colorinlistoftodos]{todonotes}
\usepackage{float}
\usepackage[font={footnotesize}]{caption}
\usepackage[numbers,sort,square,compress]{natbib}
\usepackage[para]{footmisc}
\usepackage{xcolor}
\setlength{\belowcaptionskip}{-12pt}
\newcommand{\highlight}[1]{%
  \colorbox{red!50}{$\displaystyle#1$}}

\usepackage{mathtools}
  \DeclarePairedDelimiter\floor{\lfloor}{\rfloor}
% \usepackage{parskip}
% \setlength{\parskip}{0.02\baselineskip}

% \fancypagestyle{plain}{
%   \fancyhf{} % sets both header and footer to nothing
% \renewcommand{\headrulewidth}{0pt}
%   \fancyhead[C]{2018 International Conference on Indoor Positioning and Indoor Navigation (IPIN), 24-27 September 2018, Nantes, France}% Right header

% }
\pagestyle{plain}% Set page style to plain.

% \pagestyle{fancyplain}
% \fancyhf{}
% \renewcommand{\headrulewidth}{0pt}
% \fancyhead[C]{2018 International Conference on Indoor Positioning and Indoor Navigation (IPIN), 24-27 September 2018, Nantes, France}
\DeclareMathOperator*{\argmin}{argmin}
% \usepackage{titlesec}

% \titlespacing*{\section}{0pt}{1pt plus 2p}{1ex}
% \titleformat*{\section}{\fontsize{12}{12}\bfseries}
% \titleformat{\section}
%        {\normalfont\fontfamily{phv}\fontsize{12}{17}\bfseries}{\thesection}{1em}{}

% \titleformat{\subsection}
%        {\normalfont\fontfamily{phv}\fontsize{12}{17}\bfseries\itshape}{\thesubsection}{1em}{}

\begin{document}
%Here goes the title

\title{Scalar Feedback based Joint Time-Frequency Precoder
  Interpolation for  MIMO-OFDM Systems}


%Authors List
 \author{\authorblockN{Mohit Madan, Agrim Gupta, Kumar Appaiah\\}
 \
 \authorblockA{Department of Electrical Engineering\\Indian Institute of Technology Bombay}
% \thanks{Parts of this work was supported by the Bharti Centre for Communication in
% IIT Bombay, and the Visvesvaraya
% PhD Scheme of Ministry of Electronics \& Information Technology,
% Government of India (implemented by the Digital India Corporation).
% }}
\vspace{-2.3\baselineskip}
}
\maketitle

\thispagestyle{plain}
%Main body starts



% \noindent We consider problem of quantization and interpolation of
% time and frequency varying precoding matrices in wireless MIMO
\begin{abstract}
% In a limited feedback MIMO channel, the performance of the channel can improve significantly if the transmitter knows the channel state information(CSI). The receiver knows the channel information, and it feeds back to the transmitter. However, it is not possible to feedback complete information for a limited-feedback OFDM channel since the channel information for multiple subcarriers consumes a large amount of data. Therefore to resolve this problem, the orthonormality structure of the precoder matrix is exploited to make its one-to-one mapping to a minimal number of scalar parameters. The Precoder matrix is decomposed using Givens Rotation. For an i.i.d. flat-fading Rayleigh channel, these parameters are independent, which makes its quantization and interpolation easy. We propose an efficient method for quantization of the channel state information at the receiver by using adaptive delta modulation for successive time interval. We also propose a method for joint time-frequency interpolation of the channel information at the transmitter. 153-words

  In MIMO-OFDM systems, knowledge of the precoding matrix enables
  effective resource allocation for high rates and low BER. However,
  feeding back precoders for all subcarriers introduces large
  overheads. Past work has shown that the orthonormal structure of the
  precoder enables effective parameterization and quantization using
  Givens rotations and Householder transformations. Also, the use of
  temporal and frequency correlations can further reduce the feedback
  requirements. We present a framework that utilizes adaptive feedback
  with joint time-frequency prediction for precoder construction at
  the transmitter.  Simulations reveal that the proposed method
  achieves significantly lower BERs for various channel profiles
  compared to other approaches.
\end{abstract}

\section{Introduction}
\label{intro}
% no need to write the numbers in your introduction. You start by explaining what is happening currently and then propose your solution.

% Multiple-input multiple-output (MIMO) based wireless communication
% systems that use orthogonal frequency division multiplexing (OFDM)
% have been adapted in many modern wireless systems owing to their
% flexibility and ability to use the wideband channels effectively. One
% key requirement to utilize such channels effectively is for the
% transmitter to possess channel state information (CSI), since that
% permits channel parallelization for low-complexity decoders as well as
% the use of capacity achieving transmission
% strategies~\cite{love2008overview}. This primarily translates to the
% generation of unitary precoding matrices at the transmitter. However,
% feeding back CSI from the receiver to the transmitter requires large
% feedback overhead, since a precoder needs to be fed back for each OFDM
% subcarrier within each coherence interval. Typical past approaches
% have used manifold based approaches that exploit temporal and
% frequency correlation to reduce feedback overheads. Wireless standards
% have largely employed parametric quantization that uses Givens
% rotations based quantization~\cite{lou2013comparison,5671092}, though these do
% not involve any refinement of the estimates using channel
% correlations. In this paper, we propose a joint time-frequency
% interpolation approach that builds upon the parametric quantization
% techniques used by standards that manages to outperform typical
% manifold based quantization techniques.

Multiple-input multiple-output (MIMO) based wireless communication
systems that use orthogonal frequency division multiplexing (OFDM)
have been adapted in many modern wireless systems owing to increased
data rate and channel capacity. One key requirement to achieve the channel capacity is to leverage spatial multiplexing gains obtained by precoding transmissions by Singular Value Decomposition (SVD) of the MIMO channel matrix ~\cite{love2008overview}.
However, this requires the receiver to feed back the unitary matrix obtained by SVD, to the transmitter.
Given that the channel matrix varies with both frequency and time, this feedback has to be ideally done for each subcarrier and time instance, leading to massive feedback overhead. To reduce the feedback overhead, past work taps in the inherent temporal and frequency correlations in SVD precoders, in order to perform effective quantization with just a few bits by utilizing predictive quantization \cite{Gupt1905:Predictive,6891198,7370793,sacristan2010differential,5946308,6545375,4114278,4556174}, and reducing number of fed back subcarriers by utilizing interpolation algorithms \cite{Gupt1905:Predictive,5671092,khaled2005quantized}.

% It has been shown in Agrimcite that jointly exploiting the time and frequency correlation in MIMO-OFDM systems can significantly enhance performance and reduce the feedback burden. However, using a manifold based approach complicates quantized interpolation and prediction. The use of matrix codebooks and performing arithmetic on manifolds in the time-frequency grid is significantly more complicated than in the case of scalar parameters. The approach that we suggest in this paper exploits the fact that the independent parameterization breaks the problem into several smaller scalar quantization and tracking problems. It has been shown that such approaches yield good performance for temporal  cite8 and frequency cite9 domain prediction individually. In our work, we perform prediction in the time domain, and use this for joint interpolation on the time frequency grid. We show that this allows us to exploit the time and frequency coherence jointly. We go on to show that this yields much better performance than individual time and frequency domain prediction, without the complexity of manifold based codebooks.

An underlying challenge in framing the predictive quantization and interpolation algorithms has been the high dimensional unitary structure of the SVD precoders. A large body of work utilizes the manifold geometry endowed by the unitary structure to frame such algorithms directly on the non-linear manifold ~\cite{schwarz2013adaptive,5946308,6891198,Gupt1905:Predictive,pitaval2013codebooks}. On the other hand, there are numerous works which utilize scalar parameterization of the unitary precoders, given by Givens rotations and Householder transformations \cite{4114278,4556174,lou2013comparison}. While the manifold based methods offer a clean solution and have been shown to be very effective in reducing the feedback overhead, they are analytically terse and
involve operations over higher dimensional manifolds. The scalar parameterization algorithms enable usage of the traditional algorithms to predict, interpolate and quantize over vector spaces, and hence are analytically simpler.
Such methods have also been implemented in 802.11 ac and ax standards
\cite{lou2013comparison,ieee80211}. We are not aware of performance comparisons
between these two bodies of work, thereby making it difficult to
quantify from past work whether performance gains/losses can accrue by operating over scalar parameterization instead of manifolds. In this paper, we propose a joint time-frequency interpolation approach that builds upon the existing parametric quantization techniques.
One of the key results of our work is that the proposed scalar parameterization method actually outperforms an equivalent manifold based technique \cite{Gupt1905:Predictive}, at the same time being analytically more tractable.
With smaller quantization error as compared to manifold based approaches, the feedback
results in more accurate precoder tracking, along with simple 2D-interpolation over the
scalar parameters to optimize the precoders


Past work in utilizing scalar parameterization for precoder feedback have utilized adaptive delta
modulation effectively track precoding matrices both in the
time~\cite{4114278} and frequency domains~\cite{4556174}, albeit
separately. In our work, we perform prediction in time domain, and use
this for joint interpolation on the time frequency grid. Natural
wireless channels not only have temporal and frequency correlations,
but also possess joint time-frequency correlations.  Prior work
\cite{Gupt1905:Predictive} has framed an equivalent manifold based
approach to utilize joint time-frequency correlations for predictive
quantization and interpolation, and showed performance benefits
compared to utilizing time-frequency correlations independently
\cite{6891198,khaled2005quantized}.  Intuitively, these performance
benefits arise because utilization of joint correlations not only
exploits the existing independent correlations, but also the joint
correlations in time-frequency arising from the underlying channels.

Our key contributions are two-fold.  We first propose a method for exploiting the
joint time-frequency correlation of the precoder's scalar parameters
on the transmitter using the quantized feedback.
Secondly, instead of just comparing similar algorithms in the chosen body of work (viz. scalar parameterization), as has been the case with past work, we also compare our proposed algorithm with equivalent manifold based approaches, and show signficant performance benefits over manifold approaches.


% The rest of this paper is organized as follows: Section~\ref{section2}
% describes the system model and the precoder feedback and interpolation
% techniques. Section~\ref{section3} compares the BER and achievable
% rates for our proposed methods with other approaches. Finally,
% Section~\ref{section4} concludes with discussions on future work.


\section{System Model}
For a MIMO-OFDM system with $N_T$ transmitter and $N_R$ receiver
antennas, the channel for the $i$-th subcarrier at time instant $t$
can be modeled as a flat-fading MIMO channel. This flat-fading MIMO
channel can be written as
\label{section2}
\begin{equation}
\by_{i,t} = \bH_{i,t}\tilde{\bV}_{i,t} \bx_{i,t}+ {\boldsymbol{\eta}}_{i,t}
\end{equation}
where $\bH_{i,t} \in \mathbb{C}^{N_R \times N_T}$ is the channel
matrix, $\bx_{i,t} \in \mathbb{C}^{N_d}$ is the input data vector,
$\by_{i,t} \in \mathbb{C}^{N_R}$ is the output vector,
$\tilde{\bV}_{i,t} \in \mathbb{C}^{N_T \times N_R}$ is the unitary
precoder, and $\bf{\eta}$ is the complex additive white Gaussian noise
distributed as ${\mathcal{CN}}(0,I_{N_R})$. We consider the case where
$N_R < N_T$, and the data vector size $N_d = N_R$.  The ``thin'' SVD
(that keeps only those right and left singular vectors that correspond
to non-zero singular values) of $\bH$ is given by
$\bH_{i,t} = \bU_{i,t} \Sigma_{i,t} \bV_{i,t}^{H}$, where
$\bU_{i,t} \in \mathbb{C}^{N_R \times N_R}$,
$\Sigma_{i,t} \in \mathbb{C}^{N_R \times N_R}$ and
$\bV_{i,t} \in \mathbb{C}^{N_T \times N_R}$. $\bU_{i,t}$ and
$\bV_{i,t}$ are matrices whose columns are orthonormal, and
$\Sigma_{i,t}$ contains the singular values
$\sigma_1 \geq \ldots \geq \sigma_{N_R} > 0$ of $\bH_{i,t}$. The
entries of $\bH_{i,t}$ are assumed to be distributed
i.i.d. $\mathcal{CN}(0,1)$. To obtain a precoder at the
transmitter, we need to quantize and feed back information about
$\bV_{i,t}$ for each subcarrier. However, the frequency coherence and
temporal correlations would reduce the overall feedback requirement,
as discussed in subsequent sections.

\section{Scalar joint time-frequency scheme}
\subsection{Scalar Parameterization}
\label{givens}
The degrees of freedom in the $\tilde{\bV}_{i,t}$ matrices is smaller
than the number of real entries in the matrix because of the
geometrical structure between the columns of the
matrix~\cite{4114278}.  Therefore $\tilde{\bV}_{i,t}$ could be paramaterized
using fewer parameters.
The orthonormal matrix $\tilde{\bV}_{i,t}$ with
$N_T$ rows and $N_R$ columns can be decomposed as
follows~\cite{4114278} (we suppress the $(i, t)$ subscripts in the
decomposition matrices)
\begin{equation}
\label{eq:decompose}
\tilde{\bV}_{i,t} = \left[\prod_{k=1}^{N_{R}} \bD_{k} \left( \phi_{k,k},\ldots , \phi_{k,N_{R}} \right) \:  \prod_{l=1}^{N_{T}-k} \bG_{N_{T}-l,N_{T} -l+1} \big( \theta_{k,l}\big)  \right] \: \tilde{\bI}
\end{equation}
where $\bD_{k}$ is a diagonal matrix defined as
$\bD_{k}\big(\phi_{k,k}, \ldots, \phi_{k,N_T } \big)$ =
$\mbox{diag}\big( {\bf 1}_{k-1}, e^{j\phi_{k,k}},\ldots,
e^{j\phi_{k,N_T }} \big)$ with  dimension $N_T \times N_T$. ${\bf 1}_{k-1}$ represents $k-1$ ones,
and the $N_T \times N_R$ matrix $\tilde{\bI}$ =
$\big[\bI_{N_R }, {\bf 0}_{N_R ,N_T -N_R}\big]^{T}$. Here,
${\bf 0}_{N_R ,N_T -N_R }$ is an $N_R\times (N_T - N_R)$ matrix all of
whose elements are zero. Finally $\bG_{m-1,m}\big(\theta\big)$ is a matrix with dimension $N_T \times N_T$ and is given as follows:
\begin{equation}
\bG_{m-1,m}\big(\theta\big)  =
\begin{bmatrix}
\bI_{m-2} & & & \\
& \cos\theta & -\sin\theta & \\
& \sin\theta & \cos\theta & \\
& & & \bI_{N_T -m}
\end{bmatrix}
\end{equation}
Thus, for a $4 \times 2$ orthogonal matrix $\bV_{i,t}$, we have
\begin{align*}
  \bV_{i,t} & =
  \bD_{1}(\phi_{1,1},\ldots,\phi_{1,4})\bG_{3,4}(\theta_{1,1})
  \bG_{2,3}(\theta_{1,2}) \bG_{1,2}(\theta_{1,3})\times\\
& \bD_{2}(\phi_{2,2},\phi_{2,3},\phi_{2,4}) \bG_{3,4}(\theta_{2,1}) \bG_{2,3}(\theta_{2,2})\tilde{\bI}
\end{align*}
Here, $\phi_{k,l}$s are referred to as phases, and their collection
for the $i$-th subcarrier at time instant $t$ is captured in a vector
denoted by $\bf{\Phi}$$_{i,t}$.
$\theta_{k,l}$s are called rotation angles, and their collection is
captured in a vector referred to as
$\bf{\Theta}$$_{i,t}$. We note that
$\bf{\Phi}$$_{i,t}$ and $\bf{\Theta}$$_{i,t}$ represent the phase and
rotation angles for the $i$-th subcarrier at time instant $t$, as
represented in Fig.~\ref{fig:adpm-fig}. It is known
that $\phi_{k,l} \in (-\pi, \pi]$ for all k and l's is uniformly
distributed between the two extremes~\cite{4114278}, while
$\theta_{k,l}$ is distributed as
$2l(\sin\theta_{k,l})^{2l-1}\cos\theta_{k,l}$, for
$\theta_{k,l} \in \left[0, \frac{\pi}{2}\right)$, $1\leq k \leq N_R$,
$1\leq l \leq N_T -k$.  The distribution of $\theta_{k,l}$ is used for
quantizing $\bf{\Theta}$ during initialization to reduce the
quantization error.  If $n$ bits are allocated to quantize a parameter
for the first time, the range of values that each parameter takes is
divided into $2^n$ optimal intervals using Lloyd's
quantization~\cite{lloyd1982least}. In addition, $\phi_{k,l}$ and
$\theta_{k,l}$ are statistically independent, making this
representation useful for tracking them separately across time, as
discussed below. The total number of parameters obtained from the
decomposition of a complex orthogonal matrix $N_{T} \times N_{R} $ is
$N_{R}(2N_{T} - N_{R})$. The number of $\phi_{k,l}$ parameters is
$N_{R}(2N_{T} - N_{R}+1)/2$ while the number of $\theta_{k,l}$
parameters is $N_{R}(2N_{T} - N_{R}-1)/2$. Thus, ${\bf{\Phi}}_{i,t}$
is a vector of length $N_{R}(2N_{T} - N_{R}+1)/2$, while
${\bf{\Theta}}_{i,t}$ has length
$N_{R}(2N_{T} - N_{R}-1)/2$.~\cite{4114278}
\begin{figure}
\begin{center}
\includegraphics[width=0.68\columnwidth]{images/new-adpm1.pdf}
\caption{\label{fig:adpm-fig}Subcarriers that need to be
  quantized. Subcarriers indicated by red arrows have precoders whose
  parameters are determined using the neighbouring parameters, both in
  time and frequency. The black arrow indicates refinement of past
  estimates.}
\end{center}
\end{figure}

The parameterization approach described above is related to the channel
state information feedback present in several wireless standards,
including 802.11ac~\cite{lou2013comparison}, although adaptive
quantization and feedback are not discussed. Further
sections show that the use of adaptive approaches can greatly enhance
performance with only very minor modifications to the system.

% \vspace{-0.5\baselineskip}
\subsection{Differential quantization and channel tracking}
\label{quantiz}
For channels that vary slowly with time, adaptive differential
quantization is an effective method for tracking parameters over time
using very few bits (often just one bit per parameter). For example,
while tracking a slowly varying discrete random process $x_n$ at time
$n$, the one bit can be used to determine the direction in which to
move in order to track the parameter. This can be denoted by
$\beta_{n} = \mbox{sign}(x_{n} - \hat{x}_{n-1})$, where $x_n$ is the
unquantized (accurate) value and $\hat{x}_{n-1}$ is the quantized
value. Further, the step size for the move can be adapted based on the
speed of channel variation as well the accuracy needed. If $\Delta_n$
is the step size, we define our adaptive quantizer as
\begin{equation}
\hat{x}_{n} = \hat{x}_{n-1} + \beta_{n}\Delta_{n} \mbox{ where }
\label{delta_eqn}
\Delta_{n} = \begin{cases}
    M \Delta_{n-1}, & \text{if $\beta_{n} = \beta_{n-1}$}\\
    \Delta_{n-1}/M , & \text{if $\beta_{n} \neq \beta_{n-1}$}.
  \end{cases}
\end{equation}
where $M$ is the scaling factor used during the positive and negative
moves. We initialize $\Delta_1$ as $\Delta_1 = |x_{2}-\hat{x}_1|$. This is performed separately for each of the quantized
parameters.

To quantize the subcarriers efficiently and exploit the joint
time-frequency interpolation at the transmitter, we select the
subcarriers for quantization in an alternating fashion as shown in
Fig.~\ref{fig:adpm-fig}. The figure shows the channel evolution of an
$N$ subcarrier MIMO-OFDM system with the subcarriers that are
quantized. Every $p$-th subcarrier starting with $0$ is quantized for
even time instances, where, $p$ is the gap between two quantized
subcarrier indices. For odd time instances, subcarriers at the
position $pk+q$ for $k = 0,1,..., \frac{N-1}{p}$ with $q =
{\floor*{\frac{p}{2}}}$.

For the adaptive channel tracking scheme, the arrows in
Fig.~\ref{fig:adpm-fig} show how previous quantized values along with
the 1-bit enhancement are used to find the new quantized value. Since
we use a single bit adaptation, the amount of feedback required is 1
bit for each scalar parameter that is being tracked. On average, this
can be reduced by half if we transmit only $\bf{\Theta}$ values for
one-time instance and only $\bf{\Phi}$ values for the next time
instant in an alternating fashion (the untransmitted value is assumed
to be the same as that at the previous instant). In this case, the
average number of bits required to quantize a channel matrix will be
$N_{R}(2N_{T} - N_R )/2$ (from~\cite{4114278}). While tracking
the $\phi_{k,l}$ parameters, it must be noted they may change abruptly
between $-\pi$ and $\pi$ due to jumps of $2\pi$. This is avoided by
unwrapping the phases to facilitate continuous tracking (i.e., jumps
in phase of magnitude close to $2\pi$ have been eliminated).

% \vspace{-1\baselineskip}
\subsection{Joint Time Frequency Interpolation}
\label{interp}
After obtaining the $\beta$ values (using the single bit feedback) at
the transmitter, we can construct the precoder at subcarrier indices
$pk$ or $pk+q$, keeping with the same notation as the previous
section. 
% To construct precoders for the rest of the subcarriers, we will use
% precoder parameters from both the ``past'' and ``future'' time
% instances to do the interpolation (using backtracking to ensure
% causality). When using future time instances, the implication is that the precoder
% estimation is improved at a later time instant i.e. after receiving
% the feedback of the next set of subcarriers. Note that the precoder
% used for transmission is causally obtained, while the ``future''
% correction of past precoders merely improves the interpolation
% accuracy. 

In Fig.~\ref{fig:adpm-fig}, each parameter is interpolated using the
neighbouring subcarriers using bilinear interpolation (equivalent to
linear interpolation in two dimensions). These interpolated values are
then used to reconstruct the precoding matrices all together by
applying the reverse transformation of breaking the precoder into
independent parameters as given in
Equation~\ref{eq:decompose}. Precoder estimates are calculated using
the current and past data, though the earlier precoder estimates
arexsa improved by refining the past past estimates through
interpolation with current estimates (indicated by the black arrow in
Fig.~\ref{fig:adpm-fig}). Thus, the approach here is able to exploit
the time and frequency correlation jointly to enhance performance.

% To construct the precoders for the rest of the subcarriers, we will
% use the  parameters from the previously determined subcarriers.
% 

One reason why a linear predictive quantization is preferred is
because the underlying parameters can generally be thought to emerge
from an autoregressive (AR) process. For such processes, it is well
known that linear prediction based on past samples is optimal. Even
though the $\bf{\Theta}$ and $\bf{\Phi}$ parameters do not manifest as AR
processes, for small changes, a linear approximation works
sufficiently well, as described in~\cite{4114278}, although that
consideration was limited to tracking the temporal evolution of the
parameters.

\section{Simulation and discussion}
\label{section3}
We now present simulations of the scalar feedback based joint
time-frequency predictive quantization outlined in the previous
section. In particular, we analyze the performance of the proposed
quantization and interpolation method for time-varying MIMO-OFDM
systems. The channel is simulated using the typical Jakes Model
for the COST207 channel power delay
profiles~\cite{molisch2006cost259}. The BER achieved by
the proposed predictive quantization based precoder is compared
against the ideal precoder for 100 channel evolutions in each run and
averaged over 10 channel realizations (translating to averaging over $125,000$ bits for 64-point DFT and
$2\times 10^6$ bits for the 1024-point DFT case).
The error properties obtained by simulating transmission over these
varying channels is sufficient to obtain good average performance properties.
Simulations are performed for
normalised Doppler values of $f_dT_s = 3.41\times 10^{-6}$ (corresponding
to a velocity of 37.2 km/h for $f_c = 1$ GHz) with 1024 subcarriers, and with
$f_dT_s = 1.56 \times 10^{-5}$ (velocity of 172.8 km/h) with 64
subcarriers. The sampling time period $T_s$ was $5\times10^{-8}$
s. For both the situations, we considered $N_T=4$ transmit antennas
and and $N_R=2$ receive antennas. Therefore, using the approach
outlined in~\cite{4114278}, the parameters that determine the precoder is
$N_{R}(2N_{T} - N_R) = 12$, with five $\theta_{k,l}$s and and seven
$\phi_{k,l}$s. We assume that the spacing between fed back subcarrier
indices $p$ to be $33$ and $9$ for the slow and fast channel profiles
respectively (i.e. for the slower channels with 1024 subcarriers, the
subcarriers indices fed back would be $0, 33, 66, \ldots 1023$ and for
the faster channel, with 64 subcarriers, $0, 9, 18, \ldots 63$ would
be fed back).

For the first and the second simulation time instances, since the
precoder is not available at the transmitter, we use 2 bits for
initializing each parameter for the effective representation of the
precoder. Therefore, the number of bits for each subcarrier =
$10\times 2$ = 20 (we note that $\phi_0 = \phi_4 = 0$ due to the
non-uniqueness of the SVD). Quantization is performed as given in
Section~\ref{section2}. We note that this is not significant, since,
amortized over long durations, this number becomes small. Subsequent
quantization takes place in the time-domain using one bit for each
parameter. Since the bit budget is small, if the channel varies
sufficiently slowly, the $\theta_{k,l}$ and the $\phi_{k,l}$ parameters
are fed back alternately, while retaining the unsent parameters as
their previous values, as shown in
Fig.~\ref{fig:adpm-fig}.
% Here, $\bf{\Theta}$ is the collection of all
% $\theta_l$s and $\bf{\Phi}$ is the collection of all $\phi_i$s.
This brings down the bit budget to an average of 5 bits for
$\bf{\Theta}$ and 5 bits for $\bf{\Phi}$ over alternate feedback
instances, thereby resulting in an average feedback budget of 5 bits,
as opposed to the 6 bits in~\cite{Gupt1905:Predictive}.

To enable adaptive tracking of the channel parameters, we start with
an initialization of the step size for each parameter, as discussed in
Equation~\ref{delta_eqn}. The value of $M$ was chosen to be $1.4$, and
can be adjusted appropriately to track the temporal evolution of
parameters for different Doppler frequencies.

Fig.~\ref{fig:error_decay} shows the error decay rate with time for
the proposed approach and direct quantization on the Stiefel manifold
as described in~\cite{Gupt1905:Predictive}. Since the adaptation of
independent scalar parameters causes the error to decay rapidly, the
error is lower and the convergence is faster, even for larger starting
errors in the scalar quantization approach. The independence of the
parameters enables fast convergence. The Stiefel manifold based
approach with nested codebooks may thus be suboptimal when compared to
the parameterization using angles.

\begin{figure}
\begin{center}
\includegraphics[width=\imwidth\columnwidth]{images/qerror.pdf}
\caption{\label{fig:error_decay}Comparison of the $4\times 2$
  precoder quantization error vs. time for the proposed
  approach (labelled ``Cost207\_RA Scalar'') and direct Stiefel
  manifold quantization (``Cost207\_RA Manifold'') for the COST207 channel.}
\end{center}
\end{figure}

\begin{figure}
\begin{center}
\includegraphics[width=\imwidth\columnwidth]{images/1024final_withtime}
\caption{BER vs. SNR for QPSK transmission over the $4\times 2$ 1024
  subcarrier MIMO-OFDM system, with the ITU Cost207-RA channel profile
  for $f_dT_s = 3.41\times 10^{-6}$, corresponding to a velocity of 37.8 km/h.}
\label{fig:ber_ped}
\end{center}
\end{figure}

Fig.~\ref{fig:ber_ped} compares the BER obtained using the proposed
quantization method with that obtained using exact precoders for a
$4\times 2$ 1024 subcarrier system a slow fading
($f_dT_s = 3.41 \times 10^{-6}$). Independent time-frequency based
approaches correspond to the method without joint-time frequency
interpolation i.e. with prediction in time domain along with separate interpolation in the frequency domain.
 % The time based predictive approaches
% correspond to temporal prediction using the manifold as well as scalar
% parameter based approaches without utilizing the joint time-frequency
% correlations.
We can observe that the performance of the proposed
method is close to the performance with exact precoders over a wide
range of SNRs. This can be attributed to the fact that even the 1-bit
update to the scalar parameters is able to approximate the precoder
very accurately. Since the channel variation across subcarriers is
gradual, owing to the limited delay spread of the channel profile,
linear interpolation and prediction yields good performance. Moreover,
this method outperforms the Stiefel manifold based approach since
adaptation of scalar parameters is more accurate than tangent space
based adaptation used over the Stiefel manifold. The obtained
performance moves closer to the optimal precoder when speeds reduce.
Also, similar to results shown in \cite{Gupt1905:Predictive}, the
joint time-frequency method outperforms the independent time-frequency
method, illustrating the benefits of exploiting joint correlations
versus just the independent correlations.

Fig.~\ref{fig:ber_veh} shows a similar comparison for a situation with
a higher speed. The faster variation of the channel taps with time
necessitates the use of a smaller number of subcarriers (viz. 64
subcarriers), thereby permitting more accurate tracking of precoders
on a per subcarrier basis. As in the previous case, we observe that
the scalar parameter-based approach outperforms manifold based
quantization by about 1 dB for higher SNRs. Since, on a per subcarrier
basis, the channel variation is slower, the precoder tracking is
better than the previous case. Faster channel variations would impose
limits on the efficacy of the proposed method in tracking the channel
parameters.
\begin{figure}
\begin{center}
\includegraphics[width=\imwidth\columnwidth]{images/64final_withtime}
\caption{BER vs. SNR for QPSK transmission over the $4\times 2$ 64
  subcarrier MIMO-OFDM system, with the ITU Cost207-RA channel profile
  for $f_dT_s = 1.56\times 10^{-5}$, corresponding to a velocity of 172.8 km/h.}
\label{fig:ber_veh}
\end{center}
\end{figure}


% Since the Precoding matrices fundamentally lie on the Stiefel
% manifold, therefore, we are using the chordal distance parameter to
% measure the effectiveness of the quantization method used. Later we
% are comparing the BER rates with the completely fed back
% subcarriers. Or simulations and discussions part 2 criteria could be
% used. Done similar work as in \cite{4114278} but along with that we
% have joint interpolated in time and frequency at the transmitter using
% the fundamental scalar parameters obtained. To lower the number of
% bits we have used the method of interpolation and have achieved
% significant improvement over predictive quantization method in
% \cite{6891198} and also in \cite{Gupt1905:Predictive} which tried to
% use joint interpolation over the tangent space in the Stiefel
% Manifold. In fact, we use joint interpolation of the scalar parameters
% which are easy to use and give better quantization than any of the
% above methods.

% The advantage they had over the number of bits due to the 6 bit codebook in \cite{6891198,Gupt1905:Predictive} is also achieved by us by using smart interpolation techniques, i.e. by dropping the feedback bits in alternating time instances as it is still going to follow the scalar parameters without much difference.

% The problem of values hopping between $\pi$ and $-\pi$ is also tackled by wrapping the values around which we can show works nicely. The only drawback is that while initializing the parameter, they go out of their actual range and therefore using a uniform quantizer over the prescribed range does not work. But since most of the values lie between the given range we can use most of the initialization bits quantizing the parameters uniformly between the [-$\pi$ , $\pi$] and using smaller amount of bits between the range outside that which could go up to $-3\pi$ and $3\pi$, i.e. we are covering 300% more range outside the given area and that works mostly fine for the rest of the values.

% Using the idea of joint interpolation given in the other paper.
% \vspace{-1\baselineskip}
\section{Conclusion}
\label{section4}
We have presented a predictive quantization based precoder
reconstruction for MIMO-OFDM systems that employs scalar quantization
based feedback of precoder parameters. Quantizing and adapting the
Givens rotation and Householder transformation parameters of precoders
minimizes receiver feedback to the transmitter while permitting
effective precoder adaptation. Exploiting temporal and frequency
correlations jointly enhances performance significantly. Simulations
reveal that the proposed approach converges faster and has smaller
errors when compared to manifold based approaches. Our feedback
approach, with simple 2D-interpolation over the scalar parameters,
tracks precoders effectively. Future work would involve tuning the
adaptation rates of scalar parameters and further reduction of
redundancies to improve quantization.


% \vspace{-10pt}
% \section{Acknowledgment}

% % \label {section6}

% % \input{sections/6_section.tex}

% Parts of this work was supported by the Bharti Centre for Communication in

% IIT Bombay, and the Visvesvaraya

% PhD Scheme of Ministry of Electronics \& Information Technology,

% Government of India, being implemented by Digital India Corporation.





\renewcommand{\bibfont}{\footnotesize}

\bibliography{IEEEabrv,main}

\bibliographystyle{IEEEtran}

\end{document}
