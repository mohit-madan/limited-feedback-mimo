\documentclass[12pt]{letter}
\usepackage[left=1in,twoside=false]{geometry}
\usepackage[cmex10]{amsmath}
\usepackage{amssymb}
\signature{Jinesh C Jacob}
\address{From \\
Mohit Madan\\
Department of Electrical  Engineering\\
Indian Institute of Technology Bombay, Mumbai\\
India\\
E-mail: mohit.madan@iitb.ac.in
\date{\today}
}

\begin{document}
\begin{letter}{To\\
Prof. Xiao Li\\
Editor\\
IEEE Wireless Communications Letters}
\vspace{0.5in}


\opening{Dear Prof. Li,}

Subject: \textbf{Revised submission of manuscript title ``Scalar
  Feedback based Joint Time-Frequency Precoder interpolation for  MIMO-OFDM Systems'' (Paper Number: WCL2020-0511).}

Authors: Mohit Madan, Agrim Gupta, Kumar Appaiah\\
Email: mohit.madan@iitb.ac.in

Thank you for arranging the prompt reviews of our original manuscript
submitted to the IEEE Wireless Communications Letters, and providing
us an opportunity to resubmit this manuscript with modifications
addressing the issues raised during the reviews. I would like to thank
the reviewers for a careful reading of the manuscript, as well as for
the insightful points that they have raised regarding various
technical aspects of the manuscript. Based on these reviews, we have
made revisions to our submission to address the reviewers concerns. In
particular, we have addressed the key issue raised by reviewer 2 on
the aspect of improving past precoder estimates, and have clarified
the persisting confusion. We hope that you will reconsider the
manuscript for publication in the IEEE Wireless Communications
Letters. Please find attached our comments and actions taken based on
the reviews, along with a document indicative of all the changes made.

Please do not hesitate to contact me if I may be of any assistance; I
can be reached by e-mail at mohit.madan@iitb.ac.in. Thank you again
for your consideration of this manuscript.


Thank you.
\vspace{0.3in}

Sincerely,

Mohit Madan



\end{letter}

\newpage


\textbf{Reviewer 1}\\

\textbf{My concerns have been addressed not very thoroughly, but
  acceptably. The paper is not good at theorizing the issues at hands,
  but nevertheless reports an important case
+of assembling existing techniques to achieve better performance than
theory-driven approaches do.
}

We appreciate the reviewer's comments. We have strived to make things
more clear with minor changes in the updated manuscript, while
ensuring that we adhere to the page limit. We hope that these changes
would present our idea better.

\textbf{Reviewer 2}\\

\textbf{It is still not clear how the transmitter can obtain the
  future precoder parameters.
  Could you use an example to explain it?}

We appreciate the reviewer's concern. We feel that the word ``future''
is incorrect, and have clarified in the manuscript that past estimates
are improved as parameters come in. Here is an example.

Consider the 10th OFDM frame. All precoder parameters are calculated
using the parameters obtained at the 10th, 9th, 8th ... frames to fill
in the missing precoders. While at this process, the information
obtained during the 10th precoder can improve the precoder estimates
for the 9th, 8th etc. precoders. Naturally, these ``past'' precoder
parameters cannot be used directly for precoding, but the reduced
error that they have can improve the prediction and interpolation for
the precoders in the 11th frame. It is in this spirit that we ``walk
back'' to improve past estimates so that their contribution to future
predictions would possess less error. We have now clarified this in
the manuscript as well.
\end{document}
